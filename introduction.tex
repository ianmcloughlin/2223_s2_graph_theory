\documentclass{iansnotes}

\title{Graph Theory}
\author{ian.mcloughlin@atu.ie}
\date{\today}

\begin{document}

\maketitle
An overview of graph algorithms from a computational perspective covering basic graph theory and its applications.


\section{Learning Outcomes}
On completion of this module the learner should be able to\footnote{We recently updated these learning outcomes. They are making their way through our quality assurance procedures.}:

\begin{enumerate}
  \item Define and discuss the fundamental concepts of graph theory.
  \item Model a computational problem using concepts from graph theory.
  \item Solve a computational problem using established graph algorithms.
  \item Justify the use of graph algorithms in solving a computational problem.
\end{enumerate}


\section{Book}
The main book we will use in the module is Sipser's\autocite{sipser}.
We will also rely on online documentation and sources.


\section{Assessment}
Assessment is through a final written exam and a project\footnote{The project will be in the form of a Jupyter notebook in a GitHub repository.}:

\begin{description}
  \item[$50\%$] Project
  \item[$50\%$] Final exam
\end{description}

 
\section{Delivery}
\begin{itemize}
  \item This is a semester-long module. Realistically, we will have ten uninterrupted teaching weeks\footnote{Each semester typically has thirteen teaching weeks but some of those weeks have public holidays and other interruptions.}.
  \item There are many ideas about how lecturers should deliver modules. Some suggest a top-down, structured approach where topics are clearly defined ahead of time. Others suggest involving students in decisions, letting content evolve during the semester. Let's not be idealistic about it: we'll have an initial plan and tailor it during the semester.
  \item It is worth discussing what you as a class would like to work on and giving me feedback on it. Just keep in mind that everyone might want something different. Also, remember that there is one lecturer and dozens of students in each of several modules. Time is limited, we will have to be careful about scope creep.
\end{itemize}

 
\section{Lectures and Practicals}
\begin{itemize}
  \item Traditionally, lectures covered theory and students applied the theory in practicals. That can feel a bit contrived and artificial, especially in computing where practice often comes before theory.
  \item We won't make a clear distinction between lectures and practicals where possible. Rather, we will focus on topics, projects, and problems.
\end{itemize}


\section{Topics}
We will start with a plan to cover these four topics.

\begin{description}
  \item[Structures and Operations:] Sets, tuples, binary operations.
  \item[Graphs:] Graphs, adjacency matrices.
  \item[Isomorphism:] Maps, bijections, permutations.
  \item[Automata:] Finite automata, regular languages.
\end{description}


\section{Advice}
\begin{itemize}
  \item Everyone procrastinates, you need a strategy to compensate. You will be less stressed if you work regularly, a bit every week.
  \item Review the project marking scheme regularly and work to it.
  \item Be able to demonstrate your work. This is easier for practical work, you often have code and the like. Theoretical work can be demonstrated through writing, images, plots, and diagrams.
\end{itemize}


\section{Policies}
\begin{marginfigure}%
  \centering
  \includegraphics[width=0.6\linewidth]{img/atu-green.png}
  \caption*{GMIT is now ATU.}
  \label{fig:atulogo}
\end{marginfigure}

\begin{itemize}
  \item In April 2022, GMIT merged with IT Sligo and LyIT to become ATU, the Atlantic Technological University.
  \item Although the merger has happened, it will take a couple of years for our systems and policies to fully merge.
  \item During this time, we will continue to use GMIT's policies where an ATU policy has not yet superseded them.
  \item That means the GMIT Quality Assurance Framework~\autocite{gmitqaf}.
\end{itemize} 


\end{document}